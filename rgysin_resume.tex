%%%%%%%%%%%%%%%%%%%%%%%%%%%%%%%%%%%%%%%%%
% Plasmati Graduate CV
% LaTeX Template
% Version 1.0 (24/3/13)
%
% This template has been downloaded from:
% http://www.LaTeXTemplates.com
%
% Original author:
% Alessandro Plasmati (alessandro.plasmati@gmail.com)
%
% License:
% CC BY-NC-SA 3.0 (http://creativecommons.org/licenses/by-nc-sa/3.0/)
%
% Important note:
% This template needs to be compiled with XeLaTeX.
% The main document font is called Fontin and can be downloaded for free
% from here: http://www.exljbris.com/fontin.html
%
%%%%%%%%%%%%%%%%%%%%%%%%%%%%%%%%%%%%%%%%%

%----------------------------------------------------------------------------------------
%	PACKAGES AND OTHER DOCUMENT CONFIGURATIONS
%----------------------------------------------------------------------------------------

\documentclass[a4paper,12pt]{article} % Default font size and paper size

\usepackage{xunicode,xltxtra,url,parskip} % Formatting packages
\usepackage[usenames,dvipsnames]{xcolor} % Required for specifying custom colors
\usepackage[left=.5in,right=.5in,top=.4in,bottom=.4in]{geometry}
\usepackage[none]{hyphenat}
\usepackage{multirow}

\usepackage{array}
\newcolumntype{L}[1]{>{\raggedright\let\newline\\\arraybackslash\hspace{0pt}}m{#1}}
\newcolumntype{C}[1]{>{\centering\let\newline\\\arraybackslash\hspace{0pt}}m{#1}}
\newcolumntype{R}[1]{>{\raggedleft\let\newline\\\arraybackslash\hspace{0pt}}m{#1}}

\renewcommand{\familydefault}{\sfdefault}

\usepackage{titlesec} % Used to customize the \section command
\titleformat{\section}{\Large\scshape\raggedright}{}{0em}{}[\titlerule] % Text formatting of sections
\titlespacing{\section}{0pt}{3pt}{3pt} % Spacing around sections

\newcommand\rightleftbox[1]{ \parbox{.3\textwidth}{#1} }
\newcommand\centerbox[1]{ \parbox{.4\textwidth}{#1} }

\begin{document}

\pagestyle{empty} % Removes page numbering

\font\fb=''[cmr10]'' % Change the font of the \LaTeX command under the skills section

%----------------------------------------------------------------------------------------
%	NAME AND CONTACT INFORMATION
%----------------------------------------------------------------------------------------

\par{\centering{\Huge Ryan R. Gysin}\par} % Your name
\par{\centering{gysin.ryan@gmail.com | 317 18th Ave E. Apt 3, Seattle, WA, 98112 | 989-450-1867}\par}

%----------------------------------------------------------------------------------------
%	EDUCATION
%----------------------------------------------------------------------------------------

\section{Education}

\begin{tabular}{R{2.1cm}|p{15.5cm}}
\hspace{4pt}\textsc{Apr} 2017 & Bachelor of Science in \textsc{Computer Engineering} \\
& \normalsize\emph{University of Michigan}, Ann Arbor \\
& \footnotesize{Relevant Classes: Operating Systems, Machine Learning,
  Microprocessor Design, Embedded Control Systems, Computer Security,
  Logic Design, Computer Organization, Signals and Systems}\\
& \footnotesize{GPA: 3.0/4.0} \\
\end{tabular}

%----------------------------------------------------------------------------------------
%	WORK EXPERIENCE
%----------------------------------------------------------------------------------------

\section{Experience}

\begin{tabular}{R{2.1cm}|p{15.5cm}}
\emph{Current} & \textsc{Microsoft}, Redmond, WA \\
\textsc{Nov 2018} & \emph{Software Engineer} \\
& \footnotesize{Contributed to the Windows build system} \\
& \footnotesize{Wrote unit tests for creation of Windows installable media} \\
\end{tabular}

%------------------------------------------------

\begin{tabular}{R{2.1cm}|p{15.5cm}}
\textsc{Oct 2018} & \textsc{Nexteer Automotive}, Saginaw, MI \\
\textsc{July 2017} & \emph{Manufacturing IT Engineer} \\
& \footnotesize{Designed C\# applications to act as interface
  between PLCs and SQL databases} \\
\textsc{Aug 2016} & \footnotesize{Wrote LabVIEW VIs to decode JSON messages
  and transmit them through TCP sockets}\\
\textsc{May 2016} & \footnotesize{Developed PLC routines to communicate with C\#
  application and make decisions about whether or not a part meets specifications} \\
& \footnotesize{Co-led C\# development training session specializing in WPF and
 .NET frameworks}\\
& \footnotesize{Maintained computers running on plant floor to reduce down time
  of plant lines} \\
\end{tabular}

%------------------------------------------------

\begin{tabular}{R{2.1cm}|p{15.5cm}}
\textsc{Apr 2017} & \textsc{Michigan Autonomous Aerial Vehicle (MAAV)}, Ann Arbor, MI\\
\textsc{Sept 2015} & \emph{President and Navigation Lead 2016-2017} \\
& \footnotesize{Led team of 40 to place 2\textsuperscript{nd} in the 2016
  International Aerial Robotics Competition (IARC)} \\
& \footnotesize{Developed computer vision code for detecting corners and
  ground robots based on size and color} \\
& \footnotesize{Designed and tested code that tuned computer vision software to
  reduce noise in images} \\
& \footnotesize{Organized team structure and led weekly meetings to ensure all
  sub-teams were on track} \\
& \footnotesize{Acquired corporate sponsorship totaling \$40k and managed annual
  budget} \\
& \footnotesize{Managed and reviewed entire team code base using git} \\
\end{tabular}

%----------------------------------------------------------------------------------------
%	PROJECTS
%----------------------------------------------------------------------------------------

\section{Projects}

\begin{tabular}{R{2.1cm}|p{15.5cm}}
\hspace{4pt}\textsc{Apr 2017} & \textsc{MGoKart} \\
\textsc{Jan 2017} & \footnotesize{Created autonomous gokart as concept for
 autonomous formula car} \\
& \footnotesize{Developed path planning algorithms and simple kalman filter in
  Python to steer the kart and filter out erroneous data from sensor suite,
  including a lidar and encoders} \\
& \footnotesize{Designed and built hardware architecture to allow power
 distribution and communication between central microprocessor, motors, and
 sensors} \\
& \footnotesize{Wrote code in C, Python, and Arduino to allow communication
 between the software algorithms, controls algorithms, and motors} \\
& \footnotesize{Reduced electromagnetic interference in wires across the kart by
 approximately 80\%} \\
& \footnotesize{Simulated sensor inputs to system and validated outputs using
 vehicle dynamics model} \\
\end{tabular}

%----------------------------------------------------------------------------------------
%	SKILLS
%----------------------------------------------------------------------------------------

\section{Additional}

\begin{tabular}{ll}
Languages: & C++, C, C\#, Python, T-SQL, \LaTeX, Verilog, Ruby \\
Tools: & Git, Matlab, LabVIEW, OpenCV \\
\multicolumn{2}{l}{Assisted in research resulting in childrens book about
  cavitation bubbles}\\
\multicolumn{2}{l}{Drove U of M blue buses for 3 years while in college}\\
\end{tabular}

\clearpage % Force Page Break
%------------------------------------------------

\end{document}
